% ======================================================================
% DOCUMENT METADATA - REQUIRED FOR ACCESSIBILITY
% ======================================================================
% This MUST be the first thing in your document, before \documentclass
% It configures the PDF for accessibility compliance
\DocumentMetadata{
  pdfstandard=A-2u,    % PDF/A-2u format (archival standard with Unicode support)
  lang=en-US,               % Document language (required for screen readers)
  tagging=on                  % Enable PDF tagging (required for accessibility)
}

% CRITICAL REQUIREMENTS:
% - LaTeX Engine: LuaLaTeX strongly recommended (this template requires it)
%   * pdfLaTeX has partial support but requires manual MathML files
%   * XeLaTeX does NOT support accessibility tagging
% - TeX Distribution: TeX Live 2023 or later with all packages updated
%   * TeX Live 2022 or earlier will NOT work
%   * Update using TeX Live Manager (Windows) or TeX Live Utility (Mac)
% - Overleaf: YES, this works on Overleaf via Overleaf Labs program
%   * Opt in at: https://www.overleaf.com/labs/participate
%   * Enable "Rolling TeX Live releases"
%   * Select "Rolling TeXLive (labs)" in project settings
%   * Set Compiler to LuaLaTeX
\documentclass{article}

% ======================================================================
% PACKAGE IMPORTS
% ======================================================================

\usepackage{versionPO}

\usepackage[margin=1in]{geometry}%
\usepackage{fontspec}       % Font selection for LuaLaTeX (provides \setmainfont, \setsansfont)
\usepackage{unicode-math}   % Unicode math fonts (provides \setmathfont + automatic MathML)
\usepackage{graphicx}       % Graphics support (for \includegraphics)
\usepackage{booktabs}       % Professional-quality tables
\usepackage{xcolor}         % Color support (needed for accessibility-friendly colors)
\usepackage{fancyvrb}       % Enhanced verbatim environment
\usepackage[normalem]{ulem} % Underline support (normalem prevents emphasis from being underlined)

% ======================================================================
% ACCESSIBILITY CONFIGURATION
% ======================================================================
% Configure PDF tagging for accessibility compliance

% Enable automatic alt text generation for math equations
% This helps screen readers interpret mathematical content
\tagpdfsetup{math/alt/use}

% ======================================================================
% COLOR DEFINITIONS
% ======================================================================
% Define colors with sufficient contrast for WCAG 2.1 Level AA compliance
% All colors should have at least 4.5:1 contrast ratio with white background

\colorlet{AccessibleRed}{red!80!black}      % Darker red for better contrast
\colorlet{AccessibleGreen}{green!40!black}  % Darker green for WCAG AA compliance (4.5:1 contrast)
\definecolor{AltYellow}{RGB}{255,245,170}   % Soft yellow for highlighting code

% ======================================================================
% CUSTOM COMMANDS FOR TEXT COLORS
% ======================================================================
% Shortcuts for commonly used colors in teaching materials
% Use these to maintain consistent, accessible colors throughout your slides

\newcommand\tb[1]{\textcolor{blue}{#1}}                       % Blue text
\newcommand\tr[1]{\textcolor{AccessibleRed}{#1}}     % Red text (accessible)
\newcommand\tg[1]{\textcolor{AccessibleGreen}{#1}} % Green text (accessible)
% Note: Plain yellow fails WCAG contrast requirements - use AltYellow for backgrounds only

% ======================================================================
% CUSTOM COMMANDS FOR HIGHLIGHTED CODE
% ======================================================================
% \AltBox highlights code snippets with yellow background
% Useful for drawing attention to required accessibility changes in examples

\newcommand{\AltBox}[1]{%
  {\setlength{\fboxsep}{0pt}%
   \colorbox{AltYellow}{\texttt{#1}}}%
}

% ======================================================================
% HYPERLINK CONFIGURATION
% ======================================================================
% Links should be both colored AND underlined for accessibility
% Color alone is not sufficient (fails WCAG guidelines)

\usepackage{url,hyperref}
\urlstyle{same}  % URLs use the same font as surrounding text

% Custom commands for underlined hyperlinks
\newcommand{\hrefu}[2]{\href{#1}{\uline{#2}}}  % Hyperlink with underline
\newcommand{\urlu}[1]{\uline{\url{#1}}}        % URL with underline

% Set link colors (same color for consistency)

\hypersetup{
  colorlinks=true,
  linkcolor=AccessibleGreen,   % Internal links (TOC, references)
  urlcolor=AccessibleGreen,    % External URLs
  citecolor=AccessibleGreen    % Citations
}

% ======================================================================
% DOCUMENT INFORMATION
% ======================================================================
% Define title, author, and date for article class

\title{\Large \textbf{Accessible Documents in LaTeX} \\ \normalsize \urlu{https://github.com/rhstanton/accessible_LaTeX} \\ \ \\ \large Version 1.1}
\author{Richard Stanton \\ \small UC Berkeley \\ \hrefu{mailto:richard.stanton@berkeley.edu}{richard.stanton@berkeley.edu}}
\date{\today}

% ======================================================================
% BEGIN DOCUMENT
% ======================================================================

\begin{document}

\maketitle

\section{Introduction}

On January 8, 2026, we were notified by campus that, beginning in April 2026,
  \begin{quotation}
    \color{blue}
``The updated requirements of the ADA require that digital course materials provided to students, even materials inside password-protected course sites like bCourses, will need to comply with accessibility standards (Web Content Accessibility Guidelines (\hrefu{https://www.w3.org/WAI/standards-guidelines/wcag/}{WCAG}) 2.1 Level AA).''
\end{quotation}
Many of us use LaTeX/Beamer to create teaching slides and other course materials. But Beamer is not \textbf{and never will be}  compatible with these requirements. Fortunately, the \LaTeX\ Project team has created \texttt{ltx-talk}, a \hrefu{https://latex3.github.io/tagging-project/}{purpose-built accessible presentation class} that:
\begin{enumerate}
\item Generates accessible output meeting WCAG 2.1 Level AA standards,
\item Requires little modification to existing Beamer source (uses same \texttt{frame} syntax), and
\item Requires \emph{no} manual processing of the resulting PDF file.
\end{enumerate}
\begin{itemize}
\item Part of the \hrefu{https://latex3.github.io/tagging-project/}{LaTeX Tagging Project}
\item Requires LaTeX kernel 2025-11-01 (update packages using TeX Live Manager or TeX Live Utility)
\item See \hrefu{https://latex3.github.io/tagging-project/documentation/usage-instructions}{latest detailed instructions}
\end{itemize}

\section{What is this project?}

This project began as an experiment exploring the \LaTeX\ Tagging Project's \texttt{ltx-talk} class
  \begin{itemize}
  \item \tg{Contains:} Sample slides with math, text, graphics, and tables
  \item \tg{Scores:} Perfect 100\% from the bCourses accessibility checker (Ally)
  \end{itemize}
\textbf{How to use it:}
  \begin{enumerate}
  \item \textbf{As a sample:} Working example you can copy and adapt for your slides
  \item \textbf{As documentation:} The slides and README explain what you need to do, plus heavily commented code
  \item \textbf{For quick migration:} Copy the preamble to your Beamer files for (almost) plug-and-play conversion
    \begin{itemize}
    \item \tr{Still required:} Manual tagging of figures (\texttt{alt=\{...\}}) and tables (\texttt{header-rows})
    \end{itemize}
  \item \textbf{For guidance:} Advice on accessibility considerations to keep in mind
  \end{enumerate}
Available at: \urlu{https://github.com/rhstanton/accessible_LaTeX}
  


\section{Important workflow change: Use LuaLaTeX}
You'll need to change your LaTeX compiler from \hrefu{https://www.latex-project.org/help/documentation/pdftex-a.pdf}{pdfLaTeX} to \hrefu{http://www.luatex.org/}{LuaLaTeX}.
\begin{enumerate}
\item \textbf{Automatic MathML:} LuaLaTeX automatically generates \hrefu{https://www.w3.org/Math/}{MathML} (Mathematical Markup Language), making math accessible to screen readers without extra work.
\item \textbf{Easier workflow:} While pdfLaTeX has partial support, it requires manually providing MathML files for each equation—tedious and error-prone.
\end{enumerate}
\textit{Also provides modern font support and full UTF-8 compatibility.} \\
\textbf{How to switch:}
    \begin{itemize}
    \item Command line: \texttt{lualatex myfile.tex}
    \item Most LaTeX editors: Select ``LuaLaTeX'' from the compiler menu
    \end{itemize}
\textbf{TeX version requirements:}
    \begin{itemize}
    \item \tg{Minimum:} TeX Live 2023 or later with all packages updated
    \item \tg{Critical:} Must have LaTeX kernel 2025-11-01 (update via TeX Live Manager/Utility)
    \item \tg{Overleaf:} Use Rolling TeXLive (labs) - see next slides for setup options
    \item \tr{Will NOT work:} TeX Live 2022 or earlier
    \end{itemize}

\section{You \textbf{CAN} use Overleaf (with special setup)}

\texttt{ltx-talk} requires a very recent TeX Live version. This is available through Overleaf's \tb{Labs program} (not standard Overleaf). To use this:
\begin{enumerate}
\item \textbf{Join Overleaf Labs}: 
  \begin{itemize}
  \item Visit \urlu{https://www.overleaf.com/labs/participate}
  \item Opt in and enable ``Rolling TeX Live releases''
  \end{itemize}
\item \textbf{Configure project}:
  \begin{itemize}
  \item Set TeX Live version to \tg{``Rolling TeXLive (labs)''} (bottom of list)
  \item Set Compiler to \tg{LuaLaTeX}
  \end{itemize}
\end{enumerate}
\textbf{Resources}: \urlu{https://docs.overleaf.com/writing-and-editing/creating-accessible-pdfs}



\section{The basics}

\begin{itemize}
  \item Slides are put inside a \texttt{frame} environment, just like in Beamer.
  \item So \textbf{\color{blue}existing source files don't need a lot of editing.}
  \item Here's some gratuitous  $\mathit{math}$ for the accessibility checker.
  \end{itemize}

\section{Figures}

When including a figure, you need to provide a \tr{text description}:\\

\noindent \fbox{%
  \ttfamily\small
  \textbackslash includegraphics[height=.4\textbackslash textheight,%
  \AltBox{alt=\{A capybara\}}%
  ]\{capybara.jpg\}%
}

\begin{figure}[htbp]
  \begin{center}
\includegraphics[height=.5\textheight, alt={A capybara}]{capybara.jpg}
\end{center}
\caption{A picture of a capybara}
\end{figure}

\section{Tables}

When including a table, you need to specify \tr{header rows}.  Use \texttt{\{1\}} for 1 header row, \texttt{\{1,2\}} for 2 rows, etc. E.g.,\\
    
\noindent \fbox{%
\begin{minipage}{0.95\linewidth}
\ttfamily\small
\AltBox{\textbackslash tagpdfsetup\{table/header-rows=\{1,2,3\}\}} \\
\textbackslash begin\{tabular\}\{ccccrcccr\} \\
\textbackslash toprule \\
\textrm{\color{blue}\small $\leftarrow$ 3 header rows} \\
\textbackslash midrule \\
\textrm{\color{AccessibleGreen}\small $\leftarrow$ data rows} \\
\textbackslash bottomrule \\
\textbackslash end\{tabular\}
\end{minipage}
}

\begin{table}[htbp]
\begin{center}\footnotesize
\tagpdfsetup{table/header-rows={1,2,3}}
  \begin{tabular}{ccccrcccr}
  \toprule
  \color{blue} & \color{blue} & \color{blue} & \color{blue} & \multicolumn{1}{c}{\color{blue}Days} & \color{blue}Days in & \color{blue} & \color{blue} & \color{blue}\\
\color{blue}Payment & \color{blue}Caplet & \color{blue} & \color{blue}Forward & \multicolumn{1}{c}{\color{blue}to} & \color{blue}accrual & \color{blue} & \color{blue}\\
  \color{blue}date & \color{blue}expiry date & \color{blue}$DF_\text{pay}$ & \color{blue}rate & \multicolumn{1}{c}{\color{blue}expiry} & \color{blue}period & \color{blue}$T_\text{expiry}$ & \color{blue}$\Delta$ & \multicolumn{1}{c}{\color{blue}Caplet} \\
\midrule
  \color{AccessibleGreen}2004/12/01 & \color{AccessibleGreen}--- & \color{AccessibleGreen}0.99550 & \color{AccessibleGreen}0.01790 & \color{AccessibleGreen}0 & \color{AccessibleGreen}91 & \color{AccessibleGreen}0.00000 & \color{AccessibleGreen}0.25278 & \color{AccessibleGreen}--- \hspace*{0.15in}\\
  \color{AccessibleGreen}2005/03/01 & \color{AccessibleGreen}2004/11/29 & \color{AccessibleGreen}0.99008 & \color{AccessibleGreen}0.02188 & \color{AccessibleGreen}89 & \color{AccessibleGreen}90 & \color{AccessibleGreen}0.24384 & \color{AccessibleGreen}0.25000 & \color{AccessibleGreen}1,178.77\\
  \color{AccessibleGreen}2005/06/01 & \color{AccessibleGreen}2005/02/25 & \color{AccessibleGreen}0.98401 & \color{AccessibleGreen}0.02413 & \color{AccessibleGreen}177 & \color{AccessibleGreen}92 & \color{AccessibleGreen}0.48493 & \color{AccessibleGreen}0.25556 & \color{AccessibleGreen}4,844.73\\
  \color{AccessibleGreen}2005/09/01 & \color{AccessibleGreen}2005/05/27 & \color{AccessibleGreen}0.97733 & \color{AccessibleGreen}0.02675 & \color{AccessibleGreen}268 & \color{AccessibleGreen}92 & \color{AccessibleGreen}0.73425 & \color{AccessibleGreen}0.25556 & \color{AccessibleGreen}10,016.71\\
  \bottomrule
\end{tabular}  
\end{center}
\caption{A table}
\end{table}

\section{Common pitfalls}
  \begin{itemize}
  \item \tr{Forgetting alt text for images}
    \begin{itemize}
    \item Every \texttt{\textbackslash includegraphics} needs an \texttt{alt=\{...\}} parameter
    \item Even decorative images need alt text (use \texttt{alt=\{decorative\}})
    \end{itemize}
    
  \item \tr{Not specifying table header rows}
    \begin{itemize}
    \item Add \texttt{\textbackslash tagpdfsetup\{table/header-rows=\{...\}\}} before each table
    \item Use \texttt{\{1\}} for 1 header row, \texttt{\{1,2\}} for 2 header rows, etc.
%   \item Must list \emph{all} header rows — \texttt{\{1\}} won't work if header spans 2 rows!
%    \item Forget this and screen readers can't navigate your tables properly
    \end{itemize}
    
  \item \tr{Insufficient color contrast}
    \begin{itemize}
    \item WCAG 2.1 requires 4.5:1 contrast ratio for normal text
    \item Avoid light colors: \texttt{yellow}, \texttt{cyan} fail contrast requirements
    \item Darken \texttt{red} and \texttt{green}: use \texttt{red!80!black}, \texttt{green!40!black}
    \item Standard \texttt{blue} is fine and meets WCAG requirements
    \item Test with a contrast checker: \urlu{https://webaim.org/resources/contrastchecker/}
    \end{itemize}

  \item \tr{Using the wrong compiler}
    \begin{itemize}
    \item Make sure your editor is set to use \tg{LuaLaTeX}, not \tr{pdfLaTeX}
%    \item Check this \emph{every time} you open the project
    \end{itemize}
    
  \item \tr{Old TeX distribution}
    \begin{itemize}
    \item TeX Live 2022 or earlier won't work
    \item Update packages using TeX Live Manager (Windows) or TeX Live Utility (Mac)
    \end{itemize}
    
  \end{itemize}


\section{Getting started}
\textbf{Follow these steps to migrate your Beamer slides:}
  
  \begin{enumerate}
  \item Use Overleaf Labs (see setup slide) OR install TeX Live locally
  \item If local: Update all packages using TeX Live Manager (Windows) or TeX Live Utility (Mac)
  \item Download this template from \urlu{https://github.com/rhstanton/accessible_LaTeX}
  \item Copy the preamble (everything before \texttt{\textbackslash begin\{document\}}) to your Beamer files
  \item Add \texttt{alt} text to images and \texttt{table/header-rows} to tables
  \item Set your compiler to LuaLaTeX
  \item Compile and test!
  \end{enumerate}
Questions or suggestions? \hrefu{mailto:richard.stanton@berkeley.edu}{richard.stanton@berkeley.edu}

\end{document}

% ======================================================================
% EMACS/AUCTEX LOCAL VARIABLES (Optional - for Emacs users only)
% ======================================================================
% The section below configures Emacs/AUCTeX to use the correct LaTeX engine
% and options when compiling this file from within Emacs.
%
% If you use Emacs/AUCTeX: These settings ensure LuaLaTeX is used automatically
% If you use a different editor: You can safely ignore or delete this section
%
% Key settings explained:
%   - TeX-engine: luatex           -> Forces LuaLaTeX (STRONGLY RECOMMENDED!)
%   - TeX-command-extra-options    -> Adds -interaction=nonstopmode flag
%   - TeX-master: t                -> This file is its own master document
%   - eval: commands               -> Custom Emacs functions for this file
%
% IMPORTANT: This template requires LuaLaTeX for automatic MathML generation
% Using pdflatex will require manual MathML files (tedious and error-prone)
%
% For other editors (TeXShop, TeXworks, Overleaf, VS Code, etc.):
% Configure them manually to use LuaLaTeX with -interaction=nonstopmode

%%% Local Variables:
%%% mode: latex
%%% TeX-engine: luatex
%%% TeX-command-extra-options: "-interaction=nonstopmode"
%%% TeX-master: t
%%% eval: (setq-local TeX-command-list (cons '("LatexMk (build)" "latexmk -lualatex -interaction=nonstopmode -file-line-error -synctex=1 -shell-escape -outdir=build %s" TeX-run-TeX nil t :help "Run LatexMk with LuaLaTeX") (remove-if (lambda (x) (equal (car x) "LatexMk (build)")) TeX-command-list)))
%%% eval: (setq org-format-latex-header "\\documentclass[10pt]{RSorg}\\pagestyle{empty}")
%%% eval: '(hide-latex-preamble)
%%% eval: (outline-hide-body)
%%% End:
