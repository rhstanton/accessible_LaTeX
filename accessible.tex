\DocumentMetadata{
  pdfstandard=A-2u,
  lang=en-US,
  tagging=on
}

\documentclass[frame-title-arg]{ltx-talk}

\usepackage{booktabs,xcolor}
\usepackage{fancyvrb}
% \usepackage{enumitem,icomma}

\usepackage{xcolor}

% Define useful colors
\colorlet{AccessibleRed}{red!80!black}
\colorlet{AccessibleGreen}{green!60!black}
\definecolor{AltYellow}{RGB}{255,245,170} % soft yellow highlight

\newcommand{\AltBox}[1]{%
  {\setlength{\fboxsep}{0pt}%
   \colorbox{AltYellow}{\texttt{#1}}}%
}

\newcommand\tb[1]{\textcolor{blue}{#1}}
\newcommand\tr[1]{\textcolor{AccessibleRed}{#1}}
\newcommand\tg[1]{\textcolor{AccessibleGreen}{#1}}
\newcommand\ty[1]{\textcolor{yellow}{#1}}

% Underline links
\urlstyle{same}
\usepackage[normalem]{ulem}
\newcommand{\hrefu}[2]{\href{#1}{\uline{#2}}}
\newcommand{\urlu}[1]{\uline{\url{#1}}}


\hypersetup{
  colorlinks=true,
  linkcolor=AccessibleGreen,
  urlcolor=AccessibleGreen,
  citecolor=AccessibleGreen
    }

% Make headers of new slides level 2, to stop accessibility checker complaining
\tagpdfsetup{role / new-tag = frametitle / H2}

% Stop accessibility checker complaining about math
\tagpdfsetup{math/alt/use}

% Put page number on all but first slide
\makeatletter
\renewcommand\@framenumber{%
  \ifnum\value{page}=1\relax
    % title page: no number
  \else
    \hfill\arabic{frame}% right-aligned
  \fi
}
\makeatother

% Format slide titles: left-justified, large and blue
\EditInstance{frametitle}{header}{
  font  = \LARGE\raggedright,
  color = blue
}

% Format page number
\EditInstance{footer}{std}{
  color         = black,
  font          = \tiny,
  left-hspace   = 0.9cm,
  right-hspace  = 0.5cm,
  separator     = \hfill,
  element-order = {framenumber},
}

% Font setup. This combination ensures that numerical digits look the same in both math and text mode
\setmainfont{Lato}[
  Ligatures=TeX,
  Numbers={Lining,Monospaced}
]
\setsansfont{Lato}[
  Ligatures=TeX,
  Numbers={Lining,Monospaced}
]

\setmathfont{LeteSansMath.otf}

\title{\textcolor{blue}{Accessible Slides in LaTeX}}
\author{Richard Stanton \\ \hrefu{mailto:richard.stanton@berkeley.edu}{richard.stanton@berkeley.edu}}
\institute{\small UC Berkeley}
\date{\today}

\begin{document}

\begin{frame}{}
    \maketitle
\end{frame}
\setcounter{frame}{0}

\section{Introduction}

\begin{frame}{Introduction}

\begin{itemize}

\item On January 8, 2026, we were notified by campus that, beginning in April 2026,

  \begin{quotation}
    \color{blue}
``The updated requirements of the ADA require that digital course materials provided to students, even materials inside password-protected course sites like bCourses, will need to comply with accessibility standards (Web Content Accessibility Guidelines (\hrefu{https://www.w3.org/WAI/standards-guidelines/wcag/}{WCAG}) 2.1 Level AA).''
\end{quotation}

\item Many of us use Beamer to create teaching slides.

\item {\color{AccessibleRed}But Beamer is not \textbf{and never will be}  compatible with these requirements.}

\item Fortunately, the \LaTeX\ community is developing a \hrefu{https://latex3.github.io/tagging-project/}{Beamer replacement} that
\begin{enumerate}
\item Generates accessible output,
\item Requires little modification to existing Beamer source, and
\item Requires \emph{no} manual processing of the resulting PDF file.
\end{enumerate}
\begin{itemize}
\item See \hrefu{https://latex3.github.io/tagging-project/documentation/usage-instructions}{latest detailed instructions}
\end{itemize}

\bigskip

\item You can use {\color{blue}this presentation} as a template for modifying your own work.
  \begin{itemize}
  \item Contains math, text, graphics, and tables.
  \item Scores a perfect 100\% from the bCourses accessibility checker, Ally.
  \item \tr{It's easy!} Most of the work is just editing the preamble.

  \end{itemize}
  

\end{itemize}
\end{frame}


\begin{frame}{The basics}
  \begin{itemize}
  \item Slides are put inside a \texttt{frame} environment, just like in Beamer.
  \item So \textbf{\color{blue}existing source files don't need a lot of editing.}
  \item Here's some gratuitous  $\mathit{math}$ for the accessibility checker.
  \end{itemize}
  
\vspace*{0.25in}
    
\fbox{%
\begin{minipage}{0.95\linewidth}
\ttfamily\small
\textbackslash begin\{frame\}\{The basics\}\\
\textbackslash begin\{itemize\}\\
\textbackslash item Slides are put inside a \textbackslash texttt\{frame\} environment, just like in Beamer.\\
\textbackslash item So \textbackslash textbf\{\textbackslash color\{blue\}existing source files don't need a lot of editing.\}\\
\textbackslash item Here's some gratuitous  \$\textbackslash mathit\{math\}\$ for the accessibility checker.\\
\textbackslash end\{itemize\}\\
\textbackslash end\{frame\}
\end{minipage}
}

\vspace*{0.25in}

\begin{itemize}
\item \tb{Note:} I set the fonts in this file so that numbers, percent signs, and dollar signs look the same in both math and text mode (this really bothers me\ldots)
    \begin{itemize}
    \item {\color{AccessibleRed} Text:}  \$1234567890\%.
    \item {\color{AccessibleRed} Math:}  $\$1234567890\%.$
    \end{itemize}
  \end{itemize}
\end{frame}

\begin{frame*}{Example figure}

\vspace*{.2in}

\begin{itemize}
\item Including figures is the same as in Beamer (e.g., using \texttt{\textbackslash includegraphics}),
but you need to provide a \tr{text description}.

\vspace*{0.25in}

\fbox{%
  \ttfamily\small
  \textbackslash includegraphics[height=.4\textbackslash textheight,%
  \AltBox{alt=\{A capybara\}}%
  ]\{capybara.jpg\}%
}

\end{itemize}

\vspace*{0.25in}

\centering
\includegraphics[height=.5\textheight, alt={A capybara}]{capybara.jpg}

\end{frame*}

\begin{frame}{Example table}

 \begin{itemize}
  \item Including tables is the same as in Beamer (e.g., using the \texttt{tabular} environment), but you need to specify the \tr{title rows}. E.g,

\vspace*{0.25in}
    
\fbox{%
\begin{minipage}{0.95\linewidth}
\ttfamily\small
\AltBox{\textbackslash tagpdfsetup\{table/header-rows=\{1,2,3\}\}} \\
\textbackslash begin\{tabular\}\{ccccrcccr\} \\
$\cdots$ \\
\textbackslash end\{tabular\}
\end{minipage}
}

\end{itemize}

\vspace*{0.25in}
  
\begin{center}\footnotesize
\tagpdfsetup{table/header-rows={1,2,3}}
  \begin{tabular}{ccccrcccr}
  \toprule
  & & & & \multicolumn{1}{c}{Days} & Days in & & & \\
Payment    & Caplet & & Forward & \multicolumn{1}{c}{to} & accrual & & \\
  date &   expiry date & $DF_\text{pay}$ & rate & \multicolumn{1}{c}{expiry} & period &      $T_\text{expiry}$ &    $\Delta$ &        \multicolumn{1}{c}{Caplet} \\
\midrule
  2004/12/01 &          ---        &  0.99550 & 0.01790 & 0 &  91 & 0.00000 &  0.25278 &  --- \hspace*{0.15in}\\
  2005/03/01 & 2004/11/29 &  0.99008 &  0.02188 & 89   &  90 & 0.24384 &  0.25000 & 1,178.77\\
  2005/06/01 & 2005/02/25 &  0.98401 & 0.02413 & 177 &  92 & 0.48493 &  0.25556 & 4,844.73\\
  2005/09/01 & 2005/05/27 &  0.97733 & 0.02675 & 268 &  92 & 0.73425 &  0.25556 &10,016.71\\
  \bottomrule
\end{tabular}  
\end{center}

\end{frame}

\begin{frame}{Conclusions}
  \begin{itemize}
  \item \tr{Migrating Beamer-based materials to make them accessible is quite easy.} \\[2ex]
  \item Let me know if you have any suggestions for improvement: \hrefu{mailto:richard.stanton@berkeley.edu}{richard.stanton@berkeley.edu}
  \end{itemize}
\end{frame}



\end{document}

%%% Local Variables:
%%% mode: latex
%%% TeX-engine: luatex
%%% TeX-command-extra-options: "-interaction=nonstopmode"
%%% TeX-master: t
%%% eval: (setq org-format-latex-header "\\documentclass[10pt]{RSorg}\\pagestyle{empty}")
%%% eval: '(hide-latex-preamble)
%%% eval: (outline-hide-body)
%%% End:
