% ======================================================================
% DOCUMENT METADATA - REQUIRED FOR ACCESSIBILITY
% ======================================================================
% This MUST be the first thing in your document, before \documentclass
% It configures the PDF for accessibility compliance
\DocumentMetadata{
  pdfstandard=A-2u,    % PDF/A-2u format (archival standard with Unicode support)
  lang=en-US,               % Document language (required for screen readers)
  tagging=on                  % Enable PDF tagging (required for accessibility)
}

% ======================================================================
% DOCUMENT CLASS
% ======================================================================
% ltx-talk: A presentation document class for accessible slides
%
% What is ltx-talk?
% - Created by the LaTeX Project team as part of the LaTeX Tagging Project
% - Specifically designed to generate accessible PDFs with proper tagging
% - Purpose-built replacement for Beamer that meets WCAG 2.1 Level AA standards
% - Uses Beamer-compatible syntax (frame environment) for easy migration
% - Included in TeX Live 2023+ (no separate installation needed)
%
% The [frame-title-arg] option allows frame titles to be passed as arguments
% just like in Beamer: \begin{frame}{Title}
%
% CRITICAL REQUIREMENTS:
% - LaTeX Engine: LuaLaTeX strongly recommended (this template requires it)
%   * pdfLaTeX has partial support but requires manual MathML files
%   * XeLaTeX does NOT support accessibility tagging
% - TeX Distribution: TeX Live 2023+ (TeX Live 2024 recommended)
%   * TeX Live 2022 or earlier will NOT work
%   * If using 2023, run: tlmgr update --all
\documentclass[frame-title-arg]{ltx-talk}

% ======================================================================
% PACKAGE IMPORTS
% ======================================================================
\usepackage{booktabs}      % Professional-quality tables
\usepackage{xcolor}          % Color support (needed for accessibility-friendly colors)
\usepackage{fancyvrb}      % Enhanced verbatim environment
\usepackage[normalem]{ulem} % Underline support (normalem prevents emphasis from being underlined)

% ======================================================================
% ACCESSIBILITY CONFIGURATION
% ======================================================================
% Configure PDF tagging for accessibility compliance

% Tag frame titles as H2 headings for proper document structure
% Screen readers use heading levels to navigate documents
\tagpdfsetup{role / new-tag = frametitle / H2}

% Enable automatic alt text generation for math equations
% This helps screen readers interpret mathematical content
\tagpdfsetup{math/alt/use}

% ======================================================================
% COLOR DEFINITIONS
% ======================================================================
% Define colors with sufficient contrast for WCAG 2.1 Level AA compliance
% All colors should have at least 4.5:1 contrast ratio with white background

\colorlet{AccessibleRed}{red!80!black}      % Darker red for better contrast
\colorlet{AccessibleGreen}{green!40!black}  % Darker green for WCAG AA compliance (4.5:1 contrast)
\definecolor{AltYellow}{RGB}{255,245,170}   % Soft yellow for highlighting code

% ======================================================================
% CUSTOM COMMANDS FOR TEXT COLORS
% ======================================================================
% Shortcuts for commonly used colors in teaching materials
% Use these to maintain consistent, accessible colors throughout your slides

\newcommand\tb[1]{\textcolor{blue}{#1}}                       % Blue text
\newcommand\tr[1]{\textcolor{AccessibleRed}{#1}}     % Red text (accessible)
\newcommand\tg[1]{\textcolor{AccessibleGreen}{#1}} % Green text (accessible)
\newcommand\ty[1]{\textcolor{yellow}{#1}}                   % Yellow text

% ======================================================================
% CUSTOM COMMANDS FOR HIGHLIGHTED CODE
% ======================================================================
% \AltBox highlights code snippets with yellow background
% Useful for drawing attention to required accessibility changes in examples

\newcommand{\AltBox}[1]{%
  {\setlength{\fboxsep}{0pt}%
   \colorbox{AltYellow}{\texttt{#1}}}%
}

% ======================================================================
% HYPERLINK CONFIGURATION
% ======================================================================
% Links should be both colored AND underlined for accessibility
% Color alone is not sufficient (fails WCAG guidelines)

\urlstyle{same}  % URLs use the same font as surrounding text

% Custom commands for underlined hyperlinks
\newcommand{\hrefu}[2]{\href{#1}{\uline{#2}}}  % Hyperlink with underline
\newcommand{\urlu}[1]{\uline{\url{#1}}}        % URL with underline

% Set link colors (same color for consistency)
\hypersetup{
  colorlinks=true,
  linkcolor=AccessibleGreen,   % Internal links (TOC, references)
  urlcolor=AccessibleGreen,    % External URLs
  citecolor=AccessibleGreen    % Citations
}

% ======================================================================
% SLIDE FORMATTING - PAGE NUMBERS
% ======================================================================
% Customize page numbering: no number on title slide, right-aligned on others

\makeatletter
\renewcommand\@framenumber{%
  \ifnum\value{page}=1\relax
    % title page: no number
  \else
    \hfill\arabic{frame}% right-aligned frame number
  \fi
}
\makeatother

% ======================================================================
% SLIDE FORMATTING - FRAME TITLES
% ======================================================================
% Customize the appearance of slide titles
% Left-justified, large font, blue color for visual emphasis

\EditInstance{frametitle}{header}{
  font  = \LARGE\raggedright,  % Large, left-aligned
  color = blue                  % Blue for consistency with theme
}

% ======================================================================
% SLIDE FORMATTING - FOOTER
% ======================================================================
% Customize the footer appearance (just the page number in this template)

\EditInstance{footer}{std}{
  color         = black,
  font          = \tiny,
  left-hspace   = 0.9cm,
  right-hspace  = 0.5cm,
  separator     = \hfill,
  element-order = {framenumber},
}

% ======================================================================
% FONT CONFIGURATION (OPTIONAL)
% ======================================================================
% Font choices that ensure consistency between text and math mode
% This is OPTIONAL and purely for visual consistency - not required for accessibility
%
% These fonts (Lato and LeteSansMath.otf) are included in TeX Live 2023+
% If you get font errors, you can comment out these lines or use different fonts
%
% Why these specific settings?
% - Lining + Monospaced numbers make digits look the same in text and math
% - LeteSansMath matches the Lato sans-serif style
% - This creates visual consistency that improves readability

\setmainfont{Lato}[
  Ligatures=TeX,
  Numbers={Lining,Monospaced}  % Same number style in text and math
]
\setsansfont{Lato}[
  Ligatures=TeX,
  Numbers={Lining,Monospaced}  % Same number style in text and math
]

\setmathfont{LeteSansMath.otf}  % Sans-serif math font to match Lato

% ======================================================================
% DOCUMENT INFORMATION
% ======================================================================
% Define title, author, institution, and date
% These work just like in Beamer

\title{\textcolor{blue}{Accessible Slides in LaTeX} \\ \normalsize \urlu{https://github.com/rhstanton/accessible_LaTeX} \\ \ \\ \large Version 1.0}
\author{Richard Stanton \\ \hrefu{mailto:richard.stanton@berkeley.edu}{richard.stanton@berkeley.edu}}
\institute{\small UC Berkeley}
\date{\today}

% ======================================================================
% BEGIN DOCUMENT
% ======================================================================

\begin{document}

% ======================================================================
% TITLE SLIDE
% ======================================================================
% Create the title slide using \maketitle, just like in Beamer
% The empty {} means this frame has no title
% Reset the frame counter to 0 so numbering starts at 1 on the next slide

\begin{frame}{}
    \maketitle
\end{frame}
\setcounter{frame}{0}

\section{Introduction}

\begin{frame}{Introduction}

\begin{itemize}

\item On January 8, 2026, we were notified by campus that, beginning in April 2026,

  \begin{quotation}
    \color{blue}
``The updated requirements of the ADA require that digital course materials provided to students, even materials inside password-protected course sites like bCourses, will need to comply with accessibility standards (Web Content Accessibility Guidelines (\hrefu{https://www.w3.org/WAI/standards-guidelines/wcag/}{WCAG}) 2.1 Level AA).''
\end{quotation}

\item Many of us use Beamer to create teaching slides.
\item {\color{AccessibleRed}But Beamer is not \textbf{and never will be}  compatible with these requirements.}
\item Fortunately, the \LaTeX\ Project team has created \texttt{ltx-talk}, a \hrefu{https://latex3.github.io/tagging-project/}{purpose-built accessible presentation class} that:
\begin{enumerate}
\item Generates accessible output meeting WCAG 2.1 Level AA standards,
\item Requires little modification to existing Beamer source (uses same \texttt{frame} syntax), and
\item Requires \emph{no} manual processing of the resulting PDF file.
\end{enumerate}
\begin{itemize}
\item Part of the \hrefu{https://latex3.github.io/tagging-project/}{LaTeX Tagging Project}
\item Requires LaTeX kernel 2025-11-01 (get via \texttt{tlmgr update --all})
\item See \hrefu{https://latex3.github.io/tagging-project/documentation/usage-instructions}{latest detailed instructions}
\end{itemize}

\end{itemize}
\end{frame}

\begin{frame}{This project}

\begin{itemize}
\item You can use {\color{blue}this presentation} as a template for modifying your own work.
  \begin{itemize}
  \item Contains math, text, graphics, and tables.
  \item Scores a perfect 100\% from the bCourses accessibility checker, Ally.
  \item \tr{It's easy!} Most of the work is just editing the preamble.
    \end{itemize}

\bigskip

\item The template is available at: \\
  \urlu{https://github.com/rhstanton/accessible_LaTeX}
  
\end{itemize}
\end{frame}

\begin{frame}{\tr{\textbf{IMPORTANT}: Overleaf does \textbf{NOT} currently work}}

\begin{itemize}
\item \texttt{ltx-talk} requires LaTeX kernel \tg{2025-11-01}
\item As of February 7, 2026, Overleaf is using LaTeX kernel \tr{2025-06-01}
  \item \tb{\textbf{So you must compile locally}}
  
\bigskip

\item \textbf{\tb{Installing TeX Live (Free)}}
  \begin{itemize}
  \item \textbf{Windows}: Download TeX Live from \urlu{https://tug.org/texlive/}
    \begin{itemize}
    \item Run the installer (\textasciitilde4 GB download)
    \item Choose a TeX editor: TeXworks (included) or TeXstudio
    \end{itemize}
  \item \textbf{Mac}: Download MacTeX from \urlu{https://tug.org/mactex/}
    \begin{itemize}
    \item Install the .pkg file (\textasciitilde5 GB download)
    \item Includes TeXShop editor
    \end{itemize}
  \item \textbf{After installation}: Run \texttt{tlmgr update --all} to get latest updates
  \end{itemize}

\end{itemize}
\end{frame}


\begin{frame}{Important workflow change: Use LuaLaTeX}
  \begin{itemize}
  \item \tr{You'll need to change your LaTeX compiler from \hrefu{https://www.latex-project.org/help/documentation/pdftex-a.pdf}{pdfLaTeX} to \hrefu{http://www.luatex.org/}{LuaLaTeX}.}
    
  \item \tb{Why switch to LuaLaTeX?}
    \begin{enumerate}
    \item \textbf{Automatic MathML:} LuaLaTeX automatically generates \hrefu{https://www.w3.org/Math/}{MathML} (Mathematical Markup Language), making math accessible to screen readers without extra work.
    \item \textbf{Easier workflow:} While pdfLaTeX has partial support, it requires manually providing MathML files for each equation—tedious and error-prone.
    \item \textbf{Modern Fonts:} LuaLaTeX handles OpenType fonts (like Lato) natively, essential for proper Unicode support.
    \item \textbf{Full UTF-8:} Complete support for international characters and screen readers.
    \end{enumerate}
      
  \item \tb{How to switch:}
    \begin{itemize}
    \item Command line: \texttt{lualatex myfile.tex}
    \item Most LaTeX editors: Select ``LuaLaTeX'' from the compiler menu
%    \item Overleaf: Set compiler to ``LuaLaTeX'' in the menu
    \end{itemize}

% \item \tb{TeX version requirements:}
%     \begin{itemize}
%     \item \tg{Minimum:} TeX Live 2023 (late 2023, with \texttt{tlmgr update --all})
%     \item \tg{Recommended:} TeX Live 2024 or later
%     \item \tr{Will NOT work:} TeX Live 2022 or earlier
%     \end{itemize}
      
    
  \end{itemize}
\end{frame}

\begin{frame}{The basics}
  \begin{itemize}
  \item Slides are put inside a \texttt{frame} environment, just like in Beamer.
  \item So \textbf{\color{blue}existing source files don't need a lot of editing.}
  \item Here's some gratuitous  $\mathit{math}$ for the accessibility checker.
  \end{itemize}
  
\vspace*{0.25in}
    
\fbox{%
\begin{minipage}{0.95\linewidth}
\ttfamily\small
\textbackslash begin\{frame\}\{The basics\}\\
\textbackslash begin\{itemize\}\\
\textbackslash item Slides are put inside a \textbackslash texttt\{frame\} environment, just like in Beamer.\\
\textbackslash item So \textbackslash textbf\{\textbackslash color\{blue\}existing source files don't need a lot of editing.\}\\
\textbackslash item Here's some gratuitous  \$\textbackslash mathit\{math\}\$ for the accessibility checker.\\
\textbackslash end\{itemize\}\\
\textbackslash end\{frame\}
\end{minipage}
}

\vspace*{0.25in}

\begin{itemize}
\item \tb{Note:} I set the fonts in this file so that numbers, percent signs, and dollar signs look the same in both math and text mode (my pet Beamer peave\ldots)
    \begin{itemize}
    \item {\color{AccessibleRed} Text:}  \$1234567890\%.
    \item {\color{AccessibleRed} Math:}  $\$1234567890\%.$
    \end{itemize}
  \end{itemize}
\end{frame}

\begin{frame*}{Figures}

\vspace*{.2in}

\begin{itemize}
\item Including figures is the same as in Beamer (e.g., using \texttt{\textbackslash includegraphics}),
but you need to provide a \tr{text description}.

\vspace*{0.25in}

\fbox{%
  \ttfamily\small
  \textbackslash includegraphics[height=.4\textbackslash textheight,%
  \AltBox{alt=\{A capybara\}}%
  ]\{capybara.jpg\}%
}

\end{itemize}

\vspace*{0.25in}

\centering
\includegraphics[height=.5\textheight, alt={A capybara}]{capybara.jpg}

\end{frame*}

\begin{frame}{Tables}

 \begin{itemize}
  \item Including tables is the same as in Beamer (e.g., using the \texttt{tabular} environment), but you need to specify the \tr{header rows}.
  \item Use \texttt{\{1\}} for 1 header row, \texttt{\{1,2\}} for 2 rows, \texttt{\{1,2,3\}} for 3 rows, etc.
  \item Example (table with 3 header rows):

\vspace*{0.25in}
    
\fbox{%
\begin{minipage}{0.95\linewidth}
\ttfamily\small
\AltBox{\textbackslash tagpdfsetup\{table/header-rows=\{1,2,3\}\}} \\
\textbackslash begin\{tabular\}\{ccccrcccr\} \\
$\cdots$ \\
\textbackslash end\{tabular\}
\end{minipage}
}

\end{itemize}

\vspace*{0.25in}
  
\begin{center}\footnotesize
\tagpdfsetup{table/header-rows={1,2,3}}
  \begin{tabular}{ccccrcccr}
  \toprule
  & & & & \multicolumn{1}{c}{Days} & Days in & & & \\
Payment    & Caplet & & Forward & \multicolumn{1}{c}{to} & accrual & & \\
  date &   expiry date & $DF_\text{pay}$ & rate & \multicolumn{1}{c}{expiry} & period &      $T_\text{expiry}$ &    $\Delta$ &        \multicolumn{1}{c}{Caplet} \\
\midrule
  2004/12/01 &          ---        &  0.99550 & 0.01790 & 0 &  91 & 0.00000 &  0.25278 &  --- \hspace*{0.15in}\\
  2005/03/01 & 2004/11/29 &  0.99008 &  0.02188 & 89   &  90 & 0.24384 &  0.25000 & 1,178.77\\
  2005/06/01 & 2005/02/25 &  0.98401 & 0.02413 & 177 &  92 & 0.48493 &  0.25556 & 4,844.73\\
  2005/09/01 & 2005/05/27 &  0.97733 & 0.02675 & 268 &  92 & 0.73425 &  0.25556 &10,016.71\\
  \bottomrule
\end{tabular}  
\end{center}

\end{frame}

\begin{frame}{Common pitfalls}
  \begin{itemize}
  \item \tr{Forgetting alt text for images}
    \begin{itemize}
    \item Every \texttt{\textbackslash includegraphics} needs an \texttt{alt=\{...\}} parameter
    \item Even decorative images need alt text (use \texttt{alt=\{decorative\}})
    \end{itemize}
    
  \item \tr{Not specifying table header rows}
    \begin{itemize}
    \item Add \texttt{\textbackslash tagpdfsetup\{table/header-rows=\{...\}\}} before each table
    \item Use \texttt{\{1\}} for 1 header row, \texttt{\{1,2\}} for 2 header rows, etc.
%   \item Must list \emph{all} header rows — \texttt{\{1\}} won't work if header spans 2 rows!
%    \item Forget this and screen readers can't navigate your tables properly
    \end{itemize}
    
  \item \tr{Insufficient color contrast}
    \begin{itemize}
    \item WCAG 2.1 requires 4.5:1 contrast ratio for normal text
    \item Avoid light colors: \texttt{yellow}, \texttt{cyan} fail contrast requirements
    \item Darken \texttt{red} and \texttt{green}: use \texttt{red!80!black}, \texttt{green!40!black}
    \item Standard \texttt{blue} is fine and meets WCAG requirements
    \item Test with a contrast checker: \urlu{https://webaim.org/resources/contrastchecker/}
    \end{itemize}

  \item \tr{Using the wrong compiler}
    \begin{itemize}
    \item Make sure your editor is set to use \tg{LuaLaTeX}, not \tr{pdfLaTeX}
%    \item Check this \emph{every time} you open the project
    \end{itemize}
    
  \item \tr{Old TeX distribution}
    \begin{itemize}
    \item TeX Live 2022 or earlier won't work
    \item Run \texttt{tlmgr update --all} if using TeX Live 2023
    \end{itemize}
    
  \end{itemize}
\end{frame}

\begin{frame}{Conclusions}
  \begin{itemize}
  \item \tr{Migrating Beamer-based materials to make them accessible is quite easy.}
  
  \bigskip
  \item \tb{Next steps:}
    \begin{enumerate}
    \item Download this template from \urlu{https://github.com/rhstanton/accessible_LaTeX}
    \item Copy the preamble to your existing Beamer files
    \item Change \texttt{\textbackslash documentclass\{beamer\}} to \texttt{\textbackslash documentclass[frame-title-arg]\{ltx-talk\}}
    \item Add \texttt{alt} text to images and \texttt{table/header-rows} to tables
    \item Set your compiler to LuaLaTeX
    \item Compile and test!
    \end{enumerate}
  
  \bigskip
  \item Questions or suggestions? \hrefu{mailto:richard.stanton@berkeley.edu}{richard.stanton@berkeley.edu}
  \end{itemize}
\end{frame}

\end{document}

% ======================================================================
% EMACS/AUCTEX LOCAL VARIABLES (Optional - for Emacs users only)
% ======================================================================
% The section below configures Emacs/AUCTeX to use the correct LaTeX engine
% and options when compiling this file from within Emacs.
%
% If you use Emacs/AUCTeX: These settings ensure LuaLaTeX is used automatically
% If you use a different editor: You can safely ignore or delete this section
%
% Key settings explained:
%   - TeX-engine: luatex           -> Forces LuaLaTeX (STRONGLY RECOMMENDED!)
%   - TeX-command-extra-options    -> Adds -interaction=nonstopmode flag
%   - TeX-master: t                -> This file is its own master document
%   - eval: commands               -> Custom Emacs functions for this file
%
% IMPORTANT: This template requires LuaLaTeX for automatic MathML generation
% Using pdflatex will require manual MathML files (tedious and error-prone)
%
% For other editors (TeXShop, TeXworks, Overleaf, VS Code, etc.):
% Configure them manually to use LuaLaTeX with -interaction=nonstopmode

%%% Local Variables:
%%% mode: latex
%%% TeX-engine: luatex
%%% TeX-command-extra-options: "-interaction=nonstopmode"
%%% TeX-master: t
%%% eval: (setq-local TeX-command-list (cons '("LatexMk (build)" "latexmk -lualatex -interaction=nonstopmode -file-line-error -synctex=1 -shell-escape -outdir=build %s" TeX-run-TeX nil t :help "Run LatexMk with LuaLaTeX") (remove-if (lambda (x) (equal (car x) "LatexMk (build)")) TeX-command-list)))
%%% eval: (setq org-format-latex-header "\\documentclass[10pt]{RSorg}\\pagestyle{empty}")
%%% eval: '(hide-latex-preamble)
%%% eval: (outline-hide-body)
%%% End:
